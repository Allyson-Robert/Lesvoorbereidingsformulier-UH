%%%%%%%%%%%%%%%%%%%%%%%%%%%%%%%
% Shorthand for helping texts %
%%%%%%%%%%%%%%%%%%%%%%%%%%%%%%%
\newcommand{\documenthelp}{ 
    \helpingtext{
        Deze documentclass bevat extra uitleg bij het lesvoorbereidingsformulier. 
        Je kan deze uitleg steeds aan of uitzetten door de class options te verwijderen of terug te plaatsen.
        Wil je de donkergrijze uitleg verwijderen, verwijder dan de optie ``showextra'' van de classoptions. 
        Wil je de lichtgrijze uitleg (zoals deze) verwijderen, verwijder dan de optie ``showhelp'' van de classoptions.
        Het is aangeraden om in de finale versie minstens ``showhelp'' uit te schakelen.
    }
}
\newcommand{\helpidentificatie}{
    \helpingtext{
        Vul hier de informatie rond je stageles. 
        Alle velden worden gevuld m.b.v. commando's die zich in het hoofdbestand ``lvb-main.tex'' bevinden.
    }
}
\newcommand{\helponderwerpsection}{
    \helpingtext{
        Onderwijsvorm en studierichtingen/opleidingen: Onderwijsvorm: (buitengewoon) secundair onderwijs, deeltijds kunstonderwijs, volwassenenonderwijs, verpleegkunde (HBO5), graduaatsopleiding, professionele of academische bacheloropleiding of masteropleiding. alternatieve context.
    }
}

\newcommand{\helponderwerp}{
    \extratext{Wat is het onderwerp van de les?}
    \helpingtext{Specifieer de onderwerpen goed. Welk deel ga je juist zien?}
}


\newcommand{\helpllnken}{
    \extratext{
        Cognitief (voorkennis, leerstrategie,…), motivatie (vakbeleving, …), leermoeilijkheden.
    } 
    \helpingtext{
        Wie zijn de lerenden? Welke aandacht is er voor ‘leerzorg’ (gedrag) ifv faciliteiten?
    }
}

\newcommand{\helplkrken}{
    \extratext{
        Leerkracht, (vakervaring, voorkennis, bekommernissen, professioneel zelfverstaan),  klasgroep (groepsdynamiek, klasgrootte).
    }
    \helpingtext{
            Welke zijn de voorkennis en vaardigheden die lerenden al beheersen ivm het lesonderwerp?
    }
}

\newcommand{\helpomgvken}{
    \extratext{
        School (infrastructuur, schoolcultuur,..), situationele kenmerken (tijdstip,…).
    }
    \helpingtext{
        Materiële voorzieningen (Welke soort lokaal/gebouw – infrastructuur – materiaal is er ter beschikking?) / krijtbord, Smartboard, didactisch lokaal/
    }
}

\newcommand{\helpeindtermen}{
    \extratext{
        Welke eindtermen sluiten aan bij je les? Schrijf ze hieronder uit. Vermeld ook de URL.
    }
    \helpingtext{
        SO: geef de link naar het leerplan en geef de nummering en omschrijving van de eindtermen voor de basisvorming alsook de specifieke (SET) en/of vakoverschrijdende eindtermen (VET) indien van toepassing. 
        HO:  Noteer de domeinspecifieke leerresultaten (DLR)
    }
}

\newcommand{\helpleerplandoel}{
    \helpingtext{
        Vermeld de nummers van de leerplandoelstellingen. Het is belangrijk om de les te situeren in een groter geheel dwz het deel van het leerplan opnemen en de leerplandoelstelling die vooraf gaat en de leerplandoelstelling die daarop volgt. 
        Welke leerplandoelstellingen sluiten aan bij je les? Schrijf ze hieronder uit. Vermeld ook de URL.
        SO: geef de link naar het leerplan en geef de nummering en omschrijving van de leerplandoelstelling.
        HO: Noteer de opleidingsspecifieke leerresultaten (OLR).
    }
}

\newcommand{\helplesdoel}{
    \extratext{
        Wat moeten de lerenden als resultaat van hun leerproces bereiken? Hoe stel je vast dat de lerenden de lesdoelen bereikt hebben?
    }
    \helpingtext{
        In de lesdoelen geef je aan wat de lerenden als resultaat van hun leerproces moeten bereiken. Je formuleert je lesdoelen dan ook vanuit het perspectief van de lerenden. Formuleer maximaal vijf concrete lesdoelen per lesuur om een gerichte focus te creëren. Geef per lesdoel aan hoe je vaststelt of de lerenden het lesdoel bereikt hebben(SMART formuleren= specifiek, meetbaar, acceptabel, realistisch en tijdsgebonden). “Operationaliseer en concretiseer met andere woorden in deze rubriek de leerplandoelen door deze te vertalen naar lesdoelen.” Koppeling met de taxonomie van Bloom
    Vermeld hieronder aan welke (meta)cognitieve, psycho-motorische en sociaal-affectieve lesdoelen je werkt en nummer deze. Gebruik de code ‘C’ voor de cognitieve lesdoelen, ‘PM’ voor de psychomotorische lesdoelen en ‘SA’ voor de sociaal-affectieve lesdoelen.  Formuleer niet enkel C doelen! Rangschik de lesdoelstellingen in een logische volgorde. bvb. C1, C2, PM1, SA1.
    }
}

\newcommand{\helpbronnen}{
    \extratext{
        Welke bronnen of naslagwerken gebruik je ter voorbereiding van deze les?
    }
    \helpingtext{
        \npar Houd volgende vragen in het achterhoofd wanneer je aangeeft welke studiebronnen je als voorbereiding gebruikt hebt voor deze les. Opgelet! Afhankelijk van jouw les zijn niet alle vragen even relevant. De vragen die minder relevant zijn mag je schrappen. 
        \begin{itemize}
            \item{Welke studiebronnen of naslagwerken gebruik je tijdens deze les?}
            \item{Welke vakinhoudelijke wetenschappelijke inzichten gebruik je tijdens de les? Hoe zorg je ervoor dat deze les steunt op wetenschappelijk-correcte inzichten en informatie?}
            \item{Ga niet enkel op zoek naar nationale bronnen, maar ga ook op zoek naar internationaal op zoek naar actualiteit, didactisch wetenschappelijke tijdschriften, boeken, multimedia,..}
        \end{itemize}
    }
}

\newcommand{\helpdidactmat}{
    \extratext{
        Welke leer- en didactische materialen gebruik jij tijdens deze les ter ondersteuning van de leeractiviteiten van lerenden in de les?
    }
    \helpingtext{
        Geef aan welk materiaal je zelf ontworpen of gemaakt hebt. 
        Wat heb je gekregen van de stagementor?
        Houd volgende vragen in het achterhoofd wanneer je aangeeft welke materialen je gebruikt tijdens deze les. 
        Opgelet! Afhankelijk van jouw les zijn niet alle vragen even relevant. 
        De vragen die minder relevant zijn mag je schrappen. 
        \begin{itemize}
            \item{Zijn deze ook beschikbaar voor afwezige lerenden of wanneer lerende zelfstandig aan de slag gaan met de inhouden?}
            \item{Is er studiemateriaal nodig ter oefening, herhaling of remediëring van de leerstof? Kunnen leerlingen controleren of ze goed bezig zijn? Op welke hulp kunnen ze beroep doen?}
            \item{Welke didactische materialen gebruik je tijdens deze les? Wat mag je niet vergeten om mee te nemen naar de les, zodat deze vlot kan verlopen?}
            \item{Beschik je over het nodige materiaal om je les te geven? Zijn ze herbruikbaar en duurzaam?}
        \end{itemize}
    }
}

\newcommand{\helpinfrastructuur}{
    \extratext{
        Welke infrastructuur heb je (al dan niet) ter beschikking tijdens deze les?
    }
    \helpingtext{
        Houd volgende vragen in het achterhoofd wanneer je de infrastructuur beschrijft. 
        Opgelet! Afhankelijk van jouw les zijn niet alle vragen even relevant. 
        De vragen die minder relevant zijn mag je schrappen
        \begin{itemize}
            \item{Wat is (niet) aanwezig in het klaslokaal? Heb je gecontroleerd of alles werkt?}
            \item{Wat is jouw plan B in het geval er technische problemen zijn of bij wijziging van lokaal?}
            \item{Welke klasopstelling ga je gebruiken? Maak een schets van de klasopstelling.}
        \end{itemize}
    }
}

\newcommand{\helpveiligheid}{
    \extratext{
        Hoe garandeer je de veiligheid van de leerlingen tijdens deze les? 
        Je mag deze rubriek open laten indien deze niet relevant is voor je les.
    }
    \helpingtext{
        Houd volgende vragen in het achterhoofd wanneer je de veiligheid beschrijft. 
        Opgelet! Afhankelijk van jouw les is deze rubriek niet relevant. 
        Je mag deze rubriek open laten indien deze niet relevant is. 
        \begin{itemize}
            \item{Welke zijn de H- en P-zinnen en hoe houd je hier rekening mee?} 
            \item{Hoe garandeer je de veiligheid in het labo?} 
            \item{Heb je zicht op de veiligheidsprocedures?}
            \item{Heb je zicht op het evacuatieplan?}
        \end{itemize}
        Neem de veiligheidsinstructies mee op in het practicummateriaal voor leerlingen!
    }
}
\newcommand{\helplesfase}{
    \helpingtext{
        Wat is de naam van deze lesfase? Hoe lang duurt deze lesfase (in minuten)? 
    }
}


\newcommand{\helplesdeel}{
    \helpingtext{
        \begin{lesdeel}{
            Noteer hier de codes van de lesdoelen waaraan je werkt tijdens deze lesfase. bvb. afkorting C1 of PM2 of SA1. \\

            Noteer hier aan welke leerinhoud(en) je tijdens deze lesfase werkt
            Geef ook aan welke werkvormen, groeperingsvormen en media je per leerinhoud gebruikt. 
            Je mag hier kernwoorden gebruiken. \\
            \textbf{Leerinhouden}: begrip /korte omschrijving. \\
            \textbf{Naam ‘activerende’ werkvorm}: vb \\
            onderwijsleergesprek \\
            mix tweetal gesprek, stellingenspel,\\ 
            \textbf{Naam groeperingsvorm} : vb groepswerk, coöperatief groepswerk, klassikaal, duo, individueel,...
        }
            Omschrijf het lesverloop met een helder uitgeschreven scenario. \\
            
            Noteer hier welke krachtige onderwijs- en leeractiviteiten je organiseert. 
            Beschrijf hier duidelijk welke onderwijsactiviteiten de leraar onderneemt en welke vragen hij/zij hierbij stelt aan de lerenden. 
            Beschrijf ook duidelijk welke leeractiviteiten de lerenden ondernemen.  \\
            Schrijf in deze kolom het lesverloop gestructureerd uit met behulp van alinea’s en opsommingstekens. \\
            
            Elke alinea beschrijft een deel van de lesfase.\\
            
            Leerinhoud beschrijf je kort en je verwijst naar bijlage.   \\
            
            Acties/wending/nieuwe werkvorm geef je gestructureerd weer.  \\
            
            Maak duidelijk wat de de rol is van leerkracht (LK) en van leerling (Lln) en wat de activiteiten zijn van de leerkracht. \\
            
            Voorbeeld 
            \begin{itemize}
                \item{LK legt uit:}
                \item{LK toont ...}
                \item{Lln overlegt met ...}
                \item{Lln maakt oefening, ...}
            \end{itemize}
            
            Een onderwijsleergesprek neem je op met hoofdvragen en verwachte  antwoorden.  \\
            
            \textbf{Deze kolom kan voor jou een  spreekschema en een hulpmiddel zijn. Voor derden een handig draaiboek/handleiding van jouw les.
            Schrijf bewust met korte zinnen, vermijd een doorlopende tekst.} \\
            
            Breng eventuele accenten aan met kleur en opmaak om voldoende structuur toe te  voegen. \\
            
            Verwijs naar de juiste pagina van het leerlingenmateriaal wat je in bijlage hebt toegevoegd per lesfase.
        \end{lesdeel}%
    }
}

\newcommand{\helpbijlagen}{
    \extratext{
        Voeg hieronder de relevante bijlage(n) toe. 
        Voorbeelden van relevante bijlagen zijn bordschema, PowerPoint, lesmaterialen, leerlingen materialen… 
        Vermeld per bijlage een titel (voorbeeld: 6.1. Bordschema).
    }
}


% You can organise your lesson using the \lesfase{} command and the
% \lesdeel evnironment. The \lesfase{} command generates a title and
% should be used as such. An expected duration should be included here.
% The \lesdeel environment takes 1 argument and a body. The argument
% will be placed in the first column while the body will appear in the
% second column. Multiple such environments can be added in sequence if
% desired

\section{Schematische weergave van de les en tijdsplanning}
\lesschematitel
\helplesdeel

\begin{frame}
    \lesfase{Ontdekking en vraag (5 min)}
    \begin{lesdeel}{
        Lesdoel \ref{lesdoel:lesdoel} 
        Lesmateriaal: Demo-opstelling
    }
        De opstelling wordt getoond en geschetst.
        
        Er wordt de leerlingen gevraagd of zij de belangrijke delen van de opstelling kunnen identificeren.
        De belangrijke delen zijn:
        \startolg
            \item{Harmonische aandrijving}
            \item{Krachtmeter}
            \item{Katrol}
        \stopolg
    \end{lesdeel}%
\end{frame}
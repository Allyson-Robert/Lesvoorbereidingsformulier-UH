% This section/chapter will include everything regarding the 
% contextualisation of a particular lesson

\begingroup
\Lesdoelen

% Type the lesson subject under the following line

\begin{frame}
\allCapsPar{Lesonderwerp} 
    \helponderwerp
    \thelesonderwerp
    
% Complete the elements
\allCapsPar{Beginsituatie}
    \begin{enumerate}
        \item{Leerlingkenmerken 
            \textit{n.v.t.}
            \helpllnken 
        }
        \item{Onderwijskenmerken \helplkrken 
            \begin{enumerate}
                \item{tekst kenmerk}
            \end{enumerate}
            
        De leerlingen kennen:
            \begin{enumerate}
                \item{definities van dingen}
            \end{enumerate}
            
        De leerlingen kunnen:
            \begin{enumerate}
                \item{dingen doen}
            \end{enumerate}
        
        }
        \item{Omgevingskenmerken \helpomgvken 
            \begin{enumerate}
                \item{kenmerk}
            \end{enumerate}
        }
    \end{enumerate}
\end{frame}

% Include an enumeration of the 'eindtermen' in this section
\begin{frame}
\allCapsPar{Eindtermen en beroepskwalificaties}
(\theleerplanurl)
\helpeindtermen

    \begin{enumerate}
        \item[ET1]{tekst eindterm}
    \end{enumerate}

% Include an enumeration of the 'leerplandoelstellingen' in this section
\allCapsPar{Leerplandoelstellingen} 
\thekoepel ~- \theleerplannummer ~- \theleerplanurl
\helpleerplandoel
    \begin{enumerate}
        \item[LP1]{tekst leerplandoelstelling}
    \end{enumerate}

% Include an enumeration of the 'lesdoelen' in this section
\allCapsPar{Lesdoelen en leerrendement}
\helplesdoel

De leerlingen kunnen
    \begin{enumerate}
        \item[\namedlabel{lesdoel:lesdoel}{LD1}]{een lesdoel bereiken}
    \end{enumerate}
    
\end{frame}
\endgroup
% This section/chapter will include everything regarding the 
% contextualisation of a particular lesson

\section{Lesonderwerp, lesdoelen en leerrendement}
    \helponderwerpsection

% Type the lesson subject under the following line
\allCapsPar{Lesonderwerp} 
    \helponderwerp

% Complete the elements
\allCapsPar{Beginsituatie}
    \begin{enumerate}
        \item{Leerlingkenmerken \helpllnken 
        }
        \item{Onderwijskenmerken \helplkrken 
        }
        \item{Omgevingskenmerken \helpomgvken 
        }
    \end{enumerate}

% Include an enumeration of the 'eindtermen' in this section
\allCapsPar{Eindtermen en beroepskwalificaties}
(\theleerplanurl)
\helpeindtermen

    \begin{enumerate}
        \item[ET 1]{Geef hier een eindterm in met de overeenkomstige code}
        \item[ET 2]{Er kunnen meerdere eindtermen gevoegd worden op deze manier}
    \end{enumerate}

% Include an enumeration of the 'leerplandoelstellingen' in this section
\allCapsPar{Leerplandoelstellingen} 
\thekoepel ~- \theleerplannummer ~- \theleerplanurl
\helpleerplandoel
    \begin{enumerate}
        \item[LPD 1]{Geef hier een leerplandoelstellingen in met de overeenkomstige code}
        \item[LPD 2]{Er kunnen meerdere leerplandoelstellingen gevoegd worden op deze manier}
    \end{enumerate}

% Include an enumeration of the 'lesdoelen' in this section
\allCapsPar{Lesdoelen en leerrendement}
\helplesdoel
    \begin{enumerate}
        \item[LD 1]{Geef hier een lesdoel in met de overeenkomstige code}
        \item[LD 2]{Er kunnen meerdere lesdoel gevoegd worden op deze manier}
    \end{enumerate}
